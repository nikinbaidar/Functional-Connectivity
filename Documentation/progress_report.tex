\documentclass{article}
\input{./preample}

\begin{document}
\documentclass{article}
%% Package Declarations:
\usepackage[a4 paper, margin=1in]{geometry}
\usepackage[english]{babel}
\usepackage[utf8]{inputenc}
\usepackage{subcaption}
\usepackage{pdfpages}
\usepackage{longtable}

\usepackage{times}
\usepackage[T1]{fontenc}

\usepackage[compact]{titlesec}
\usepackage{graphicx}
\usepackage{enumitem}
\usepackage{colortbl}
\usepackage{float}
\usepackage{multirow}

\usepackage{acronym}
\usepackage{tocloft}
\usepackage{setspace}
\usepackage{url}
\usepackage[hidelinks,linktoc=all]{hyperref}
\usepackage[backend=biber,style=ieee]{biblatex}

%% Configs

\urlstyle{same}

\onehalfspacing

\definecolor{sea}{RGB}{175,255,215}

\DeclareUnicodeCharacter{0301}{\'\i}

\addto\captionsenglish{
 \renewcommand{\contentsname}
 {Table of Contents}
}

\addbibresource{./references.bib}

% \setlength{\parindent}{0em}
% \setlength{\parskip}{1.5pt}


\begin{document}

\begin{titlepage}
  \fontfamily{stix}\selectfont
  \newcommand{\HRule}{\rule{\linewidth}{0.1mm}}

  \begin{center}

    {\huge Purbanchal University} \\[0.5cm]

    {\huge College of Biomedical Engineering and
    Applied Sciences} \\[1cm]

    \includegraphics[width=0.4\textwidth]{./.img/cbeas-logo.png} \\[1cm]

    \textbf{\Large \textit{Progress Report}} \\[0.5cm]

    \color{red}
    \HRule \\[0.4cm]
    \color{black}

    {\huge \bfseries Seed-based Functional Connectivity Analysis of
    Hippocampal Network of Patients Suffering from \linebreak Major
    Depressive Disorder}\\[-0.1cm]

    \color{red}
    \HRule \\[1.2cm]
    \color{black}

    \begin{minipage}{0.4\textwidth}
      \begin{flushleft}
        \textbf{\Large \textit{ Submitted By: } \\[0.3cm]
        % Authors:
        Lucky Chaudhary [A27]   \\
        Namrata Tamang [B3]     \\
        Nikin Baidar [B4]       \\
        Nilima Sangachchhe [B5] \\
        Shashwot Khadka [B18]   \\
        Sneha Khadka [B22]      \\[2.2mm]
        Suhana Chand [B23]
        } \\ [1cm]
      \end{flushleft}
    \end{minipage}
    ~
    \begin{minipage}{0.4\textwidth}
      \begin{flushright}
        \vspace{-135pt}
        \textbf{\Large \textit{Supervisor:} \\[0.3cm]
        Er. Alaka Acharya \\
        }
      \end{flushright}
    \end{minipage}\\[2cm]

    \vfill
    \enlargethispage{\baselineskip}
    { \Large \today}

  \end{center}

\end{titlepage}

\end{document}


\begingroup
\fontsize{12pt}{11pt}\selectfont

% preface
\pagenumbering{roman}
\clearpage
\setcounter{page}{1}
\hskip180pt {\textbf{\centering \Large Preface} }
\vskip40pt
\addcontentsline{toc}{section}{Preface}

The basis for this project stemmed from the fact that not many
researches have been conducted in Nepal regarding the diagnosis of
mental disorders. Nonetheless, in other countries, many researches and
studies have been conducted regarding the functional connectivity of
different brain regions in depression and other mental
disorders. However, till date, there is no solid evidence that could
be used for the clinical diagnosis of mental disorders. Our project
intends to review past researches and keep up with the studies related
to Major Depressive Disorder and brain functional connectivity. In
addition to that, we have selected hippocampal circuity as the region
of interest for our purposes and the overall project is going to
revolve around how functional connectivity of hippocampal network in
MDD patients differ from that of healthy people.

The following progress report begins with the basic concepts of the
resting state functional connectivity of brain, different regions of
brain and MR images. The concept of MR image processing, functional
connectivity analysis will be used in the project. Making use of these
concepts, our project approaches to compare and come up with a
conclusion about how functional connectivity differs in MDD patients
and how the findings of the project could play a role in diagnosis of
MDD. In this project, the use of ``functional connectivity'' is
restricted to mean quantification of the operational interactions of
multiple spatially distinct brain regions that are not engaged in any
specific task or stimulus. We will further restrict our discussion to
connectivity measures of the hippocampal network, derived from fMRI
imaging modality alone.  We are required to conduct this project as a
part of the curriculum of Biomedical engineering. We hope to acquire a
profound knowledge about MR image processing which would help us a lot
in our career as biomedical engineers. \\ [1cm]

\hspace*{137mm}\textbf{\textit {$-$ Authors}}
\newpage

%abstract
\hskip180pt {\textbf{\centering \Large Abstract} }
\vskip2pt
  \addcontentsline{toc}{section}{Abstract}

\newpage

\thispagestyle{empty}
\addcontentsline{toc}{section}{List of Figures}
\listoffigures

\thispagestyle{empty}
\addcontentsline{toc}{section}{List of Tables}
\listoftables
\newpage

% Abbreviations

\begin{center}
  \section*{Abbreviations}
  \addcontentsline{toc}{section}{Abbreviations}
\end{center}

\begin{acronym}
  \acro{AFNI}{Analysis of Functional Neuroimages}
  \acro{BDI}{Beck Depression Inventory}
  \acro{BOLD}{Blood Oxygen Level Dependent}
  \acro{DSM}{Diagnostic and Statistical Manual of Mental Disorders}
  \acro{fMRI}{Functional Magnetic Resonance Imaging}
  \acro{HC}{Healthy Controls}
  \acro{MDD}{Major Depressive Disorder}
  \acro{MR}{Magnetic Resonance}
  \acro{ROI}{Region of Interest}
  \acro{rs}{Resting State}
  \acro{rsFC}{Resting-state Functional Connectivity}
  \acro{SCA}{Seed-based Correlation Analysis}
  \acro{sMRI}{Structural Magnetic Resonance Imaging}
\end{acronym}

\newpage
% table of contents
\vskip -10pt
\enlargethispage{\baselineskip}
\thispagestyle{empty}
\addcontentsline{}{}{}
\tableofcontents
\newpage

\pagenumbering{arabic}
\clearpage
\setcounter{page}{1}

\section{Introduction}

So, based on our objective proposed, this refers to the progress
report of our project till date along with the future works to be
performed that completely meet our aims.

\subsection{Background}

Functional MRI is well established as a method for the detection and
delineation of regions of the brain that change their level of
activation in response to specific stimulus. fMR imaging modality
is sensitive to fluctuations in the BOLD signal which reflects
neuronal activation, or neuronal activity.

The specific objective of this project is to perform seed-based
functional connectivity analysis to study how the normal function of a
particular area of a healthy brain gets disrupted in diseased
condition. Major Depressive Disorder being one of the major mental
health illness in our country, this project aims to analyze the
functional changes in the brain of MDD patients compared to HC.
Furthermore, in our literature review, it was found that the
hippocampus of the brain is one of major regions affected by a variety
of neurodegenerative and mental disorders. For this reason, we aim to
assess the functional connectivity of the hippocampal region of the
brain.

\subsection{Rationale}

The absence of biological markers makes it exceptionally difficult for
neurologists to diagnose a person with a psychiatric disorder. The
diagnostic procedures that are the gold standard for the diagnosis of
neurodegenerative and psychiatric disorders, in the present day are
wholly based on behavioral observations and patient reported symptoms,
both of which do not have a molecular or radiological basis. Although
there have been countless studies conceptualizing the possibility of
implementation of various functional imaging modalities for
deciphering the etiology and the functional affects of various mental
disorders, the finding from these studies do not appear amongst the
diagnostic criteria. A critical barrier to the clinical translation of
many such findings is the reverse inference fallacy.

The information from structural radiology modalities describe the
shape, size and integrity of brain structure, but they do not provide
any information about the brain function. Nonetheless, as we will
discuss in the coming sections of this progress report, combining
structural MRI and functional MRI can be a promising technique to
characterize normal and abnormal brain function, which can act as an
auspicious biomarker for neurodegenerative or psychiatric disorders to
determine the risk, progression and therapeutic effectiveness. This
project can lay the foundations for further research and development
on this particular field.

\section{Methodology}

\subsection{Data Acquisition}%
\label{sub:data_acquisition}

For the following research, we acquired functional MR as well as T1 MR
images of subjects from two distinct categories. The first set of
subjects are categorized as healthy controls, and the second set of
subjects are categorized as those who are suffering from major
depression. The subjects from the first category, i.e. healthy
controls will be referred to as ``HC'' and the subjects from the
second category, i.e. depressed subjects will be referred to as ``MDD
patients''. The functional and anatomical MR images for the
progression of this study were acquired from the SRPBS Multidisorder
MRI Dataset. The datasets included were obtained from the DecNef
Project Brain Data Repository, which is a repository of neurological
images gathered by a consortium as a part of the Japanese Strategic
Research Program for the Promotion of Brain Science supported by the
Japanese Advanced Research and Development Programs for Medical
Innovation (AMED). Furthermore, the MR datasets included in this
repository were diagnosed by trained and experienced neurologists,
which assures that our study will be tilted more towards accuracy and
efficiency.

\subsection{Data Selection}%
\label{sub:data_selection}

The SRBPS dataset originally had the MR images of more than 1400
volunteers. All of these volunteers have underwent a standardized
clinical evaluation protocol, which included a general and
neurological evaluation. This opts for the accuracy of our study. Out
of these 1400 volunteers, 15 right-handed subjects from each category
were manually picked. For MDD patients, only those volunteers were
selected whose BDI was greater than 30.

The age group of the selected subjects ranged from 30 to 50 and
subjects of either sex, male or female were selected. Statistical
tests were performed on the selected subjects to verify if the
subjects from both categories are well-matched based on sex as well as
age.

to accuracy as manual analysis may create more chaos.After this, those
analysed data are converted to nifti extension from DICOM followed by
the conversion from 3D to 4D MRI Images.  Meanwhile, the structural
brain MR-images are subjected to segmentation process through SPM.
Further, the functional brain MR images are preprocessed. These tasks
have been accomplished while our end goal is yet to be achieved which
is to analyze the preprocessed images on the basis of their functional
connectivity of both MDD and HC patients, comparing them and
presenting the analysis.

To meet our objective, we employed various computational tools like MS
Excel, MRIcron, Octave, SPM(Statistical Parametric Mapping), AFNI
(Analysis of Functional Neuroimages) in a LINUX Operating Environment.
All these tools are open-source softwares that added feasibility to
the progress of our objectives.


Progress of this project includes Statistical t-test analysis of the
separated data of both MDD and HC patients, deploying various
computational tools for conversions, segmentation and preprocessing of
the images till the date. First of all, literature review of papers
and articles was done which helped us boost our theoretical knowledge
and develop our own strategy to meet our project's objective.

\subsection{fMRI Data Selection and Statistical Analysis}

In this project, Data selection includes retrieving disparate types of
data from DecNef Project Brain Data Repository. We retrieved 30 MRI
images of public data set out of which 15 are MDD and 15 are HC. We,
first, preferred to take the dataset with respect to the age and sex
lying between 20 to 30 years age group but since the required dataset
was not enough as per our decided number of patients, we extracted the
patients of age group lying between 30 to 50 years. Also for gender,
we decided to take equal number of male and female data, for that we
selected equal number of male and female manually and to verify that,
we performed Chi-squared test . To perform that test, we created a
pivot table in MS Excel and then, we inserted the number of male and
female in the pivot table.Then,applying the formula of chi-squared
test in the table,we verified that the number of male and female data
are equal.Now,for the age group, we performed t-test,so, we acquired
the age of both MDD patients and HC in two rows and in MS Excel, we
applied t-test (two tailed distribution) to check the validation of
age group.

\subsubsection{Chi-squared test}

A chi-squared test is a statistical hypothesis test that is valid to
perform when the test statistic is chi-squared distributed under the
null hypothesis. Pearson's chi-squared test is used to determine
whether there is a statistically significant difference between the
expected frequencies and the observed frequencies in one or more
categories of a contingency table. The purpose of the test is to
evaluate how likely the observed frequencies would be assuming the
null hypothesis is
% true.[https://www.jmp.com/en_be/statistics-knowledge-portal/chi-
% square-test.html] So,we performed this test for verification of the
% equal
presence of manually selected number of Male and Female which resulted
out to be true.

\subsubsection{t-test}

A t-test is a type of inferential statistic used to determine if there
is a significant difference between the means of two groups, which may
be related in certain features. A t-test looks at the t-statistic, the
t-distribution values, and the degrees of freedom to determine the
statistical significance.
[https://www.investopedia.com/terms/t/t-test.as p] So, we performed
this test for the verification of the presence of desired i.e, 20 to
30 years of age group of both MDD ans HC patients. Thus, from the
large SRPBS public dataset, we extracted 15 MDD and 15 HC patients
whose age group lie between 30 to 50 age group including equal number
of male and female having BDI Index greater than 3 that indicated the
accuracy to our project's analysis.

\subsection{Data Conversion}

Next step includes data conversion of extracted one i.e, the obtained
public dataset is in DICOM format so, we basically converted those
extracted data to NIFTI format for further processing compatibility.
For this, in linux environment,we installed ,MRIcron (magnetic
resonance image conversion, viewing and analysis) was installed using
command 'sudo apt-get install' MRIcron so that we can load multiple
layers of images, create volume renderings and draw region of
interest. It also provides 'dcm2nii' tool that can convert DICOM
format images to NIFTI format.DICOM(Digital Imaging and Communications
in Medicine) is a standard, internationally accepted format used to
view, store, retrieve and share medical images. NIfTI is the format
that attempts to keep spatial orientation information. Many of the
popular tools used for scientific image processing, analysis and
visualization require images to be stored in the NIfTI file format,
whereas scanners used to acquire these images usually export data in
the DICOM format. DICOM standard is complicated and different scanner
manufacturers extend the DICOM standard in a variety of ways,often
resulting in duplication of information and incompatibilities between
software only designed to work with one particular subset of DICOM.
Therefore, while one conversion tool may work for many images, it may
fail for others. So,the DICOM format images are converted to NIfTI
format. During the installation it didn't meet the dependencies so
the dependencies were corrected using command 'sudo apt --fix-broken
install'. 'which dcm2niigui' command was used to check if the dcm2nii
tool is present or not, '/usr/bin/dcm2niigui' if appeared then it
indicates dcm2nii tool is installed. dcm2nii can be used in two ways
either by dragging and dropping files onto the program or by launching
dcm2nii from command line and specifying the options to use. We used
the drag and drop method: firstly all the DICOM format images were
placed into a folder followed by dragging and dropping of images onto
dcm2nii one by one. The File/ModifyNIfTI command can be used to change
existing NIfTI images which can guide us through selecting the images
and choose how we'd like to modify the images(remove volumes, changing
sub format, reorienting, or changing the order of the 3rd and fourth
dimension). The Neuroimaging Informatics Technology Initiative (NIfTI)
provides coordinated and targeted service, training, and research to
speed the development and enhance the utility of informatics tools
related to neuroimaging. NIfTI focuses on tools used in fMRI because
fMRI is rapidly growing, and improvement in informatics tools will
provide broad benefits for neuroscience and greater cost/benefit
ratio. NIfTI is more useful and useable neuroimaging informatics tools
that provides environment to facilitate convergence on common
solutions to widespread problems and maximizes scientific
opportunities from neuroimaging research. This format has been widely
adopted in neuroimaging research, allowing scientists to mix and match
image processing and analysis tools developed by different teams. For
slice orientation other than axial, the image could be re-organized as
if it were axial slice. This won’t cause any problem in terms of the
NIfTI format, and may avoid confusion in some analysis and
visualization tools. ( write about the script for nifti conversion
Bloody no)

\subsection{Skull stripping of rsfmri Images}

Skull stripping is the process of isolating brain tissue from
non-brain tissue from an MRI image of a
brain. [https://www.analyticsvidhya.com/blog/2021/06/introduction-to-skull-stripping-image-segmentation-on-3d-mri-images/]
Skull stripping may improve the robustness of the registration
process, since high resolution structural images contain considerable
amounts of non-brain tissue such as eyeballs, bone, skin, and other
tissues. This step shows general improvement in the quality of the
normalization and template congruence for skull-stripped images in
every tissue category (whole brain, gray matter, white matter).Skull
Stripping improves registration and normalization. Skull stripping
consists of following three steps: 1) Volume is preprocessed to
exclude gross spatial image non-uniformity artifacts the brain is
repositioned in reasonable manner. 2) A spherical surface is
iteratively expanded until the brain is enveloped. 3) Various masks
and surfaces that models the brain and portions of the skull are
created.
% [https://reader.elsevier.com/reader/sd/pii/S2213158213001241?token=F773DAB7A62EA6DF418D1A8B3EDB16A9D4579350603E8B1B4E431FFB44C3CC6143E846DA0E48A556068EF5E0ABCD5CBA&originRegion=eu-west-1&originCreation=20220216102651]

3D skull-stripping is done on T1 image(structural images of brain)
using the command: Command: 3dSkullStrip -input filename -prefix

We run this command in LINUX terminal by selecting the T1 images of
specific folders of MDD and HC patients, we isolated the brain tissue
from non-brain tissue of all T1 images of MDD and HC patients. So as
to view the isolated brain tissue of fMRI data of each patient's
brain, we viewed the skull stripped brain images on the MRIcron by
properly selecting the T1 images of each MDD ad HC subjects.


\subsection{Image Segmentation}

Segmentation is the most important part in image processing. Fencing
off an entire image into several parts which is something more
meaningful and easier for further process. These several parts that
are rejoined will cover the entire image. Segmentation may also depend
on various features that are contained in the image. It may be either
color or texture. Before de-noising an image, it is segmented to
recover the original image. The main motto of segmentation is to
reduce the information for easy analysis. Segmentation is also useful
in Image Analysis and Image Compression.{cite1}

In our project, segmentation is done to identify grey matter, white
matter and CSF from the fMRI images of the brain. So, we proceed the
segmentation process using SPM(Statistical Parametric mapping).
Segmentation in SPM can work with images collected using a variety of
sequences , but the accuracy of the resulting segmentation will depend
on the particular properties of the images.

SPM (Statistical Parametric Mapping) is an fMRI analysis software
package that is run in Matlab. So, first installed MATLAB which would
require the MATLAB licensing that would ultimately add cost to our
project. Thus, We preferred to install Octave instead of MATLAB since
Octave being open source software.

\subsubsection{Octave}

Octave is a high-level interpreted language,
primarily intended for numerical computations. It provides
capabilities for the numerical solution of linear and nonlinear
problems, and for performing other numerical experiments. It also
provides extensive graphics capabilities for data visualization and
manipulation. Octave is also known as GNU Octave.It is mainly used in
solving the linear and non linear problems numerically, and for
performing numerical experiments it is mostly compatible with MATLAB.
It mainly consists of function calls or script. Its syntax is mainly
matrix-based and provides various functions for matrix operations.
Octave does support various data structures and object-oriented
programming. It has great features and compatible with other languages
like syntax and functional compatibility for MATLAB. It shares other
features like built-in support for complex numbers, powerful built-in
math functions, and extensive function libraries and in terms of
user-defined functions as well. Octave is normally used
through its interactive command line interface, but it can also be
used to write non-interactive programs. The Octave language is quite
similar to Matlab so that most programs are easily portable.

\subsubsection{SPM12}

SPM(Statistical Parametric Mapping) refers to the
construction and assessment of spatially extended statistical
processes used to test hypotheses about functional imaging
data.{cite0}SPM runs in the octave package. SPM requires the image
data in a suitable format. Here, data used is in nifti format. There
are two main forms of nifti – here used form is ``.nii''files , which
combines all the information into one file. Nifti images can be
single 3D volumes, but 4D or even 5D images are possible. VBM
Pre-processing in SPM12 include: Use of Segment for characterising
intensity distributions of tissue classes, and writing out
``imported'' images that DARTEL can use. Run DARTEL to estimate all
the deformations.

\subsubsection{Image Segmentation Process}

Segmentation is done to identify grey and white matter. Grey matter
contains high densities of unmyelinated (lacking a myelin sheath)
neurons, white matter contains high densities of myelinated neurons
and CSF is contained in large amount in ventricles. Segmentation in
SPM can work with images collected using a variety of sequences, but
the accuracy of the resulting segmentation will depend on the
particular properties of the images. Although multiple scans of each
subject were available, the dataset to be used only includes the
T1-weighted scans. For greater accuracy of registration, the tissues
of our anatomical image need to be accurately mapped to the tissues of
template. Segmentation can be found within SPM-Spatial-Segment in the
batching system. The output of segmentation will be used for achieving
more accurate inter-subject alignment using DARTEL. In this project,
data used were a selection of T1-weighted scans of healthy
control(HC).

\subsubsection{Image Segmentation Algorithms}

There are various techniques used in image segmentation.following are
some methods:

1) Region based segmentation Region based segmentation
is of two types:

a) Threshold Segmentation: In this segmentation technique, image gray
scale information processing is directly divided based on the gray
value of various targets. Segmentation effect can be obtained if the
target and background have high contrast. following are the types of
Threshold segmentation:

i) local threshold method: In this method multiple segmentation
thresholds are selected and images are divided into multiple target
regions and backgrounds using multiple thresholds.

ii) Global threshold method: In this method image is divided into two
regions of target and background using a single threshold.

b) Regional growth segmentation: Firstly a
seed pixel is selected and eventually similar pixels are merged around
seed pixel into the region, where seed pixel is located.

2)Edge detection segmentation:

A gray edge between two adjacent regions with different gray values is
always present in the image and case in which the gray value is not
continuous. Those discontinuities are detected using derivative
operations(derivatives are calculated using differential operators).
Generally, Parallel edge detection is used as a technique of image
preprocessing, which is done by means of a spatial domain differential
operator so as to perform image segmentation by convolution of its
template and image.

3)Segmentation based on clustering:

In this technique, the pixels in image space are segmented with
corresponding feature space, which is then segmented and mapped back
to original image space so as to generate segmentation outcome.
k-means is the most common clustering algorithm used. It basically
gathers samples into different clusters based on the distance. For the
achievement of compact and independent clusters as clustering targets
the two points need to closer.

As we are mainly concerned with the grey matter, white matter and CSF
of HC fMRI Images, we chose the tissues associated to grey matter,
white matter and CSF only along with the respective native tissues
being selected as "Native+DARTEL". After everything had been set up,
the image segmentation starts. As a result, there should be a bunch of
new images files generated. Files containing ``c1'' in their name are
what the algorithm identifies as grey matter. If they have a ``c2''
then, they are supposed to be white matter. The `` c3'' images, are
CSF.

\subsubsection{DARTEL}

DARTEL (Diffeomorphic Anatomical Registration Through Exponentiated
Lie Algebra) The idea behind DARTEL is to increase the accuracy of
inter-subject alignment by modeling the shape of each brain using
millions of parameters. DARTEL works by aligning grey matter among the
images, while simultaneously aligning white matter. This uses the
imported ``rc1'' and ``rc2'' images, and generates ``u\_rc1'' files,
as well as a series of template images. Main idea of DARTEL is to
register images by computing a flow field which can then be
exponentiate to generate both forward and backward deformations .We
need DARTEL because it allows an accurate inter-subject registration
of brain images which is necessary for both proper separation of
different segment , different tissue classes and after it gets
specially normalized towards common global MNI we will be able to do
group wise statistics and to extrapolate those findings to result of
other studies. Processing begins with the ``import'' step. This
involves taking the parameter files produced by the segmentation, and
writing out rigidly transformed versions of the tissue class images,
such that they are in as close alignment a possible with the tissue
probability maps. The next step is the registration itself. This
procedure begins by creating a mean of all the images, which is used
as an initial template. Deformations from this template to each of the
individual images are computed, and the template is then re-generated
by applying the inverses of the deformations to the images and
averaging. This procedure is repeated a number of times. After
running DARTEL by selecting the previously generated c1, c2, c3 images
of all the fMRI images of HC,the file names beginning with ``r'' (as
in ``rc1'') are the DARTEL imported versions of the tissue class
images, which will be aligned together next.

\subsection{Image Preprocessing}

Structural radiology modalities such as an MRI or a CT only provides
information about brain anatomy. Nonetheless, nuclear MRI provides
enough information about the structure of the brain that complements
functional MRI in a number of ways. Since, the brain function
ultimately depends on the integrity of the brain structure, the
underlying tissue integrity allows one to examine the functional
signals. Essentially, structural MRI provides an anatomical reference
for visualization of activation patterns and regions of interest to
extract functional connectivity information.

fMRI image preprocessing is basically
done to improve the quality of the image so as to analyse it in a
better way. Preprocessed images can suppress undesired distortions
and enhance some features which are necessary for the particular
application we are working for. An fMRI volume contains not only the
signal that we are interested in, changes in oxygenated blood but also
fluctuations that we are not interested in, such as head motion,
random drifts, breathing, and heartbeats.These other fluctuations are
noise, which are to be separated from the images or its region that we
are interested in.

fMRI images are of lower resolution because they are collected at a
very fast rate. So, they are needed to be

1) reconstructed
2) renamed,
3) made into 3D bricks
4) reregistered (to correct for head movement),
5) deconstructed
6) formatted into another type of image file, and
7) normalize

\subsubsection{AFNI}

Analysis of Functional Neuro Images is an open-source environment for
processing and displaying functional MRI data which is a technique for
mapping human brain activity. AFNI is a set of programs that can be
used interactively or flexibly assembled for batch processing using
shell script. It is developed for the analysis and display of multiple
MRI modalities such as anatomical, functional MRI (FMRI) and diffusion
weighted data. It is freely available for research purposes. AFNI
allows researchers to overlay a functional image to the anatomical,
providing tools for aligning the two into the same space.

The AFNI performs the following pre-processing steps and finish along
with a basic regression analysis:

1.Slice timing: Each 3D brain image is composed of multiple 2D images
or slices. Although acquired at approximately the same time, up to
several seconds could separate the first slice acquired from the last.
Through interpolation, the slices are aligned to the same time point.
Generally, any introduced noise from interpolation errors is thought
to be outweighed by improvements in signal.

2.Motion correction: Head movements can create sources of error in the
analysis. Each 3D acquisition in a scan is collected on a 3D grid,
with each small cube of grid space which is also called voxel.
Representing a single image intensity value. Ideally, voxels will
always represent the same part of the brain in each acquisition,
rather than vary from one 3D image to the next. To correct small
motion artifacts, AFNI' s motion correction tool employs a linear
least squares algorithm that attempts to align each 3D image acquired
to the first image acquired in the scan.

3.Smoothing: To reduce random noise in the image, a smoothing is
applied. While smoothing can increase the signal-to-noise ratio of
the image, it reduces image resolution.


4.Mask: Removes any non-brain areas, such as skull, from
the fMRI image. 5.Scale: Scale each voxel so that changes in
intensity represent percentage of signal change over the course of the
scan. The default sets the mean of each voxel equal to 100.

To initialize the preprocessing steps sequentially, we run various
scripts as indicated

1) Prep-2-BOLD-EC-preproc.sh
2) Prep-3-BOLD-EC-align-to-anat-MNI1.sh
3) Prep-7-reCovariates

All these scripts contains various commands that are used in image
preprocessing in AFNI.

1) Prep-2-BOLD-EC-preproc.sh

3dcalc - What it does- This program does voxel-by-voxel arithmetic on
3D datasets. Why we used it- To convert DICOM files to AFNI format
and then exclude the first 4 TRs.

3dDespike - What it does- It removes 'spikes' from the 3D+time input
dataset and writes a new dataset with the spike values replaced by
something more pleasing to the eye.

3dToutcount -automask: It calculates number of 'outliers' a 3D+time
dataset, at each time point. So it is used to count the outliers at
each TRs. Using base command, TR with less is detected and use that TR
with least outliers as base for head motion correction and spatial
normalization.

3dvolreg- The AFNI command to perform motion correction and estimate
spatial deviations between the reference functional image and other
functional images using each of the movement parameters like base,
tshift

3dAutomask- It is used for the brain masking of the obtained subject.

3dcalc- It helps to maskout the functional BOLD outside of the brain.



2) Prep-3-BOLD-EC-align-to-anatomical-MNI1.sh This scripts includes
following command that are run on the

3dUnifize - What it does- Bascially white matter in T1-weighted images
is made reasonably uniform in intensity,through the brain regions.
Why we used it-we use the 3dUnifize program to (approximately)
spatially uniformize and normalize intensities throughout the brain
region, which helps in the matching process, especially when using
datasets from different scanners.

@Align\_Centers- We've used @Align\_Centers twice in the script.
1st)Moves the center of DSET to the center of BASE.BASE: Base volume,
typically a template. DSET is typically an anatomical dset to be
aligned to BASE. Options used: -cm : Center is the center of mass of
the volume. Why we used it- to roughly move the center of the base
image of BOLD data with the anatomical data of each subject.
2nd)roughly moves the center of the aligned anatomical data with
standard anatomical template in the TT\_icmb space. Options used:
-child: A bunch of datasets, originally in register with DSET, that
should be shifted in the same way.

align\_epi\_anat.py- This Python script computes the alignment between
two datasets, typically an EPI and an anatomical structural dataset,
and applies the resulting transformation to one or the other to bring
them into alignment. Options used: -volreg off : to not perform
volume registration on EPI dataset before alignment -tshift off : to
not perform time shifting of EPI dataset before alignment -anat2epi :
align anatomical dataset to EPI dataset (default) master\_anat:
-master grid resolution for anatomical to epi output -epi\_base: Base
sub-brick to use for alignment.Choose sub-brick number or statistic
type(for example, 0,5,mean) -anatomical\_has\_skull no :anatomical is
assumed to have no skull epi\_strip:method to mask brain in EPI data.
Why we used it- to linearly align the anatomical data to BOLD data.

ln - This makes link between files.

3dAllineate- It is a program to align one dataset (the 'source') to a
'base' dataset, using an affine (matrix) transformation of space.
Options used: -cmass = Use the center-of-mass calculation to determin
an initial shift [This option is OFF by default] -autoweight = Compute
a weight function using the 3dAutomask algorithm plus some blurring of
the base image. lpa- it allows autoweight -source\_mask sss = Mask
the source (input) dataset, using 'sss


3dQwarp- This program computes a nonlinearly warped version of
source\_dataset to match base\_dataset. Why we used it? - To produce
a dataset warped to match the TTicbm template. Since the I152 template
is already somewhat blurry, the amount of blurring applied to it is
set to zero, and also the source dataset will be Gaussian blurred with
a FWHM of 0 mm. After using 3dQwarp he source dataset is warped to
match the base and gets prefix as output.


3dCopy - This program will copy all datasets using the old\_prefix to
use the new\_prefix.

3dNwarpApply - Program to apply a nonlinear 3D warp saved from 3dQwarp
to a 3D dataset, to produce a warped version of the source dataset.
Options used: -master mmm = 'mmm is the name of the master dataset,
which defines the output grid. -nwarp option has two filenames inside
single quotes,this feature tells that program to compose (catenate)
those 2 spatial transformations before applying the resulting warp.
Why we used it? -to apply the nonlinear transformation and resample to
3x3x3 mm

3drefit -This program changes some of the information inside a 3D
dataset's header. Options used: -view code- Changes the 'view' to be
'code', where the string 'code' is one of 'orig', 'acpc', or 'tlrc'.
Why we used it?- to convert orig view to tlrc view.

3dDeconvolve - Program to calculate the deconvolution of a measurement
3D+time dataset with a specified input stimulus time series. Options
used: [-polort pnum]- -polort option allows the use of 'A' to set the
polynomial order automatically.   The purpose of '-polort' is to
build the columns of the regression matrix. Why we used it? 1-to get
tissuse-based signal before smooth. 2-linear detrending

3dBandpass-This program is similar to 3dFourier. This is a program to
lowpass and/or highpass each voxel time series in a dataset. Options
used: -band fbot ftop = Alternative way to specify passband
frequencies. fbot = lowest frequency in the passband, in Hz ftop =
highest frequency in the passband (must be > fbot) -retrend -any mean
and linear trend are removed before filtering.

3dmerge - This program has 2 different functions: (1) To edit 3D
datasets in various ways (threshold, blur, cluster, ...); (2) To merge
multiple datasets in various ways (average, max, ...). Either or both
of these can be applied. Options used: -doall = Apply editing and
merging options to all sub-bricks uniformly in a dataset. Why we used
it? -For spatial smoothing. Specifically :to apply a 6mm FWHM(Full
Width at Half maximum) Gaussian blur

3) Prep-7-reCovariates


(left)


\section{Results and Discussions}

The methods mentioned above are brought into implementation stepwise
and its results are mentioned amd discussed below.

\subsection{Data Selection and Conversion}

The total fMRI data of 15 MDD and 15 HC of age group between 20 to 50
years with equal number of male and female were accurately obtained
and both the chi square test and t-test resulted out to be true and
those dataset were further converted to the nifti extension from DICOM
format so at to establish the compatibility in AFNI environment.


( add picture of ms excel and nifti conversion what?)

\subsection{Skull Striping}

By running the command of 3dskullstrip, the
scans of skull were removed from the structural images i.e, T1 images
of all the subjects. The skull stripped structural images of all
subjects were viewed in MRIcron so as to conform the error free
command execution.


( skull stripped images)

3.4 nikin Image Segmentation Running the HC subjects for segmentation
process, results the formation of rc1 images for grey matter, rc2
images for white matter and rc3 images for CSF that will be used for
the common masking. (segmented images)

3.5 nikin Scripts Preprocessing

\section{Conclusion and Further Work}

In a nutshell, the extracted dataset are statistically analysed so as
to obtain the accuracy which would remove the differences that are
created from manual analysis. All the MDD and HC subjects are also
converted to NIFTI format to access in the AFNI formats from
DICOM.Also, we successfully segmented the structural images of HC that
resulted in isolation of grey matter, white matter and CSF which will
further be used in common masking. Also, the images are skull
stripped using AFNI commands to each subjects and are further used to
preprocessed so that the images would be free from noises and any
distortions by running bash scripts to each subjects.Thus these are
the tasks that are performed so far to meet our objectives and
successfully achieved our hypothetical results. Mentioning about our
further works,in a preprocessed images,statistical tests will be
implemented to for a thorough analysis of the functional connectivity
of the seed. Specifically, we plan to assess functional connectivity
between various regions of the brain and hippocampal area. To assess
functional connectivity in the brain region, Resting-state analyses,
that is, time series correlations in BOLD fMRI data acquired in a
task-free state will be used. A statistical approach to image analysis
makes it possible to discover spatial and temporal patterns that
correspond to the performance of specific tasks and specific
diagnoses. Such statistical methods have only begun to be applied to
clinical disorders but show promise for increasing the ``specificity''
of brain imaging markers for mental illness.

\section*{References}
\addcontentsline{toc}{section}{References}
\printbibliography[heading=none]

\section*{Appendix}
\addcontentsline{toc}{section}{Appendix}

\appendix

\section{Hello}



\endgroup % Font is restored to 10pt

\end{document}
